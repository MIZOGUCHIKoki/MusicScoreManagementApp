\chapter{本番環境}
\section{本番環境の設定-Rails}
本番環境への設定には,Railsを動かすアプリケーションサーバとして\texttt{Puma},
Webサーバとして\texttt{nginx}を使用する.
\subsection{Master keyの設定}
マスターキーは,pushしてもGitHubには上がらないため,EC2上で作成する.
ローカル環境でマスターキーを確認する.
\begin{screen}
    \texttt{\$ cat config/master.key}
\end{screen}
EC2にログインし,マスターキーを作成する.
\begin{screen}
    \texttt{\$ vi config/master.key}
\end{screen}
ローカル環境で取得したマスターキーの中身を記述する.

\subsection{環境変数の設定}
EC2上で\texttt{credentials.yml}を設定する.
\begin{screen}
    \texttt{\$ EDITOR="vi" bin/rails credentials:edit}
\end{screen}
ファイル内にAPIキーやDBのパスワードなどの秘密情報を必要に応じて記述する.
\texttt{credentials.yml}の環境変数の利用は以下のように行う.
\begin{screen}
    \texttt{Rails.application.credentials. + ハッシュオブジェクト名 + [:キー]}
\end{screen}
\texttt{master.key}の内容は共有すれば,\texttt{credential.yml}の内容を
他メンバーからも編集できる,他メンバーにも知られたくない個人の秘匿情報は,サーバ上
の\texttt{/etc/environment}に環境変数として設定しておく.

\subsection{nginxの設定}
\texttt{nginx}をインストールし,設定ふぁいるを編集する.
\begin{screen}
    \texttt{\$ sudo yum -y install nginx}
    \texttt{\$ vi /etc/nginx/conf.d/APPLICATION\_NAME.conf}
\end{screen}
以下を\texttt{nginx.conf}に記述する.
\begin{screen}
\begin{lstlisting}
# log directory
error_log  /var/www/rails/{APPLICATION\_NAME}/log/nginx.error.log; 
access_log /var/www/rails/{APPLICATION\_NAME}/log/nginx.access.log;

# max body size
client_max_body_size 2G;
upstream app_server {
  # for UNIX domain socket setups
  server unix:/var/www/rails/{APPLICATION\_NAME}/tmp/sockets/puma.sock fail_timeout=0; 
}
server {
  listen 80;
  server_name 自身のElasticIP;

  # nginx so increasing this is generally safe...
  keepalive_timeout 5;
  # path for static files
  root /var/www/rails/{APPLICATION\_NAME}/public;
  # page cache loading
  try_files $uri/index.html $uri.html $uri @app;

  location @app {
    # HTTP headers
    proxy_set_header X-Forwarded-For $proxy_add_x_forwarded_for;
    proxy_set_header Host $http_host;
    proxy_redirect off;
    proxy_pass http://app_server;
  }

  # Rails error pages
  error_page 500 502 503 504 /500.html;
  location = /500.html {
    root /var/www/rails/{APPLICATION\_NAME}/public; 
  }
}
\end{lstlisting}
\end{screen}
権限を変更する.
\begin{screen}
    \texttt{\$ cd /etc}
    \texttt{\$ sudo chown -R ec2-user nginx}
    \texttt{\$ ls -l}
\end{screen}

\subsection{Pumaの設定}
railsの\texttt{config/puma.rb}を編集する.
\begin{lstlisting}
    #port        ENV.fetch("PORT") { 3000 }
    bind "unix://#{Rails.root}/tmp/sockets/puma.sock"       
\end{lstlisting}

\subsection{本番環境での起動}
静的ファイルのコンパイルを行う.
\begin{screen}
    \texttt{\$ bundle exec rails assets:precompile RAILS\_ENV=production}
\end{screen}
\texttt{nginx}を起動する.
\begin{screen}
    \texttt{\$ sudo service nginx start}
\end{screen}
\texttt{Puma}を起動する.
\begin{screen}
    \texttt{\$ bundle exec rails s -e production}
\end{screen}