\documentclass{reviewSheet}
\dateTime{2023/12/10 8:00-20:00}
\room{Online}
\groupName{dir-en-gray}
\author{\mizo}
\object{内部設計書 v1}
\member{1250297,1250352,1250341,1250372,1250373,1250382,1250385}
\newcommand{\oku}{奥平 舜理}
\newcommand{\naka}{中村 祐貴}
\newcommand{\tana}{田中 諒}
\newcommand{\mika}{三上 柊}
\newcommand{\mizo}{溝口 洸熙}
\newcommand{\yama}{山田 滉希}
\newcommand{\yamat}{山本 祥弘}

\begin{document}
\begin{rev}
    1 & R & 19 & ScoresController.newフロー図の機能名が「ユーザ削除」となっている &  &  & \yamat  & 機能名を「楽譜データ登録画面表示」に変更 &  & \naka  & 12/10 \\
    2 & Q & 2/6 & ユーザ一覧閲覧機能は必要ないのかどうか &  &  & \mika  &  &  & \tana  & 12/9 \\
    3 & Q & 3/3 & 「JavaScriptが使えるブラウザ」という表現方法は正しいのか &  &  & \mika  & 「対応しているブラウザ」や「有効なブラウザ」という表現などのほうが正しいのではないか &  & \tana  & 12/9 \\
    4 & R & 4/7 & 不必要な行なのではないか &  &  & \mika  & 「指摘された箇所は,すべて確認,訂正する.」というのは制作する際の注意点であり,コーディング規約ではないと考えられる &  & \tana  & 12/10 \\
    5 & R & 40/2 & システム提案書となっている部分を内部設計に修正 &  &  & \mika  & 内部設計書に修正 &  & \tana  & 12/9 \\
    6 & R & 40/16 & 貢献内容のミス &  &  & \mika  & 「テンプレート更新」ではなく「テンプレート作成」の間違い &  & \tana  & 12/9 \\
    7 & R & 40 & \texttt{texttt}がそのまま出力されている &  & \ck & \mika  & \texttt{texttt}を適用させる &  &  \mizo  & 12/10 \\
    8 & C & 38 & 図が一枚に収まりませんかね・・・ &  &  & \mika  & 収まりませんでした &  & \mizo  & 12/10 \\
    9 & R & 40 & 貢献内容のモジュール名のミス &  &  & \mika  & 山本祥弘の貢献内容の担当モジュール名を修正 &  & \yamat  &  12/10 \\\hline
\end{rev}
\end{document}