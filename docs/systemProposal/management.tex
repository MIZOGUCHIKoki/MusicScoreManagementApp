\chapter{運用保守}
\section{運用}
運用の項目として以下のものが考えられる.
\begin{itemize}
    \item 重大な問題の発生時には管理者によりサービスの停止・再開を制御する.
    \item システムの変更時にはバックアップを取得しておく.
    \item 運用終了は,収支などを理由としての終了が予想される.
\end{itemize}
\subsection{利用申請について}
利用者は,利用申請をGoogleFormにて行う.管理者はGoogleFormで受け取った内容を確認し,利用者へID,パスワードを配布する.
\section{保守}
保守の項目として以下のものが考えられる.
\begin{itemize}
    \item サーバの運用および保守はレンタルサーバの提供会社に委託する.
    \item バグ報告や機能改善要望はメール窓口により受け付ける.
    \item セキュリティ向上のために,パッケージやアプリケーションの更新をする.
\end{itemize}
\subsection{メール窓口について}
メールは,GoogleやYahoo!などのフリーメールアドレスを作成し,受信したメッセージに対して対応する.
メールアドレスは,全ページ共通のフッタに記載する.
\subsection{バックアップ}
\figref{fig:情報の流れ}にあるとおり,保存されたデータをバックアップするデータベースの準備が必要である.
DBのバックアップは,本番環境へデプロイする前に1回,そして1日1回の定期バックアップをする.
\subsection{セキュリティの向上}
パスワードのブルートフォースアタックについては,Railsの\texttt{rack-attack}を用いて対策する.
また,漏洩したパスワードを設定しないよう促す,Railsの\texttt{pwne}を導入する.
GitHubの\texttt{dependabot}を用いて,\texttt{gem}経由で適用しているパッケージのバージョンはできるだけ最新版に更新する.