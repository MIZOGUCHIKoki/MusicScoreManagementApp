\chapter{運用保守}
\section{運用}
運用の項目として以下のものが考えられる.
\begin{itemize}
	\item 重大な問題の発生時には管理者によりサービスの停止・再開を制御する.
	\item システムの更新時には,インシデントに対して復元を可能にするためのデータを保存する.
	\item 運用終了は,収支などを理由としての終了が予想される.
\end{itemize}
\section{保守}
保守の項目として以下のものが考えられる.
\begin{itemize}
	\item サーバの運用および保守はレンタルサーバの提供会社に委託する.
	\item バグ報告や機能改善要望はメール窓口により受け付ける.
	\item セキュリティ向上のために,パッケージなどの更新をする.
\end{itemize}
\subsection{メール窓口について}
メールについては,GoogleやYahoo!などのフリーメールアドレスを作成し,受信したメッセージに対して対応する.
メールアドレスは,全ページ共通のフッタに記載する.
\subsection{バックアップ}
機能改善などに伴う新システムの適用時には,適用前のシステムをバックアップとして管理者が保存しておく.
このバックアップデータは,新システム提供時にバグが発見された際,旧システムに復元することを目的としている.
なお,データベースの保守に関しては,サーバをレンタルしていることから,管理者は保守作業を行わない.
\subsection{セキュリティの向上}
パスワードのブルートフォースアタックについては,Railsの\texttt{rack-attack}を用いて対策する.
また,漏洩したパスワードを設定しないよう促す,Railsの\texttt{pwne}を導入する.
GitHubの\texttt{dependabot}を用いて,\texttt{gem}経由で適用しているパッケージのバージョンはできるだけ最新版に更新する.