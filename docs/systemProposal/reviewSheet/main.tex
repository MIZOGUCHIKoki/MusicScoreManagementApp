\documentclass{reviewSheet}
\dateTime{2023/10/23 15:10-16:10}
\room{高田研究室}
\groupName{dir-en-gray}
\author{溝口 洸熙}
\object{システム提案書v2}
\member{1250297,1250352,1250341,1250372,1250373b,1250382,1250385}
\newcommand{\oku}{奥平 舜理}
\newcommand{\naka}{中村 祐貴}
\newcommand{\tana}{田中 諒}
\newcommand{\mika}{三上 柊}
\newcommand{\mizo}{溝口 洸熙}
\newcommand{\yama}{山田 滉希}
\newcommand{\yamat}{山本 祥弘}

\begin{document}
\begin{rev}
    1 & R & - & 「{{削除することができる}}」は冗長な表現. & & & \naka &「{{削除することができる}}」から「削除できる」に変更. & 2023/10/21 & \mizo & 2023/10/23 \\
    2 & R & - & 句読点が使われている.→「.」「,」 & & \ck & \yama & 「.」「,」に変更. & 2023/10/21 & \mizo & 2023/10/23 \\
    3 & R & - & 文末を統一してください. & & \ck & \yama & 文末を,「だ」「である」調に変更. & 2023/10/21 & \mizo & 2023/10/23 \\
    4 & R & - & 一文が長すぎる & &  & \yama & 一文を短くした. & 2023/10/21 & \mizo & 2023/10/23 \\
    5 & Q & - & 「開示」は「開始」の間違いですか.& & \ck & \tana & 変更要望を\mizo に提出. & 2023/10/21 & \mika & 2023/10/10\\
    6 & Q & - & ソースは必要ないでしょうか. & & \ck & \tana & 使用したリンクを※で追加. & 2023/10/21 & \mika & 2023/10/23\\
    7 & R & - & 箇条書き\texttt{4,5}に重複項目がある. & & \ck & \mizo & \texttt{5.}を「編集」から「削除」に変更.& 2023/10/21 & \naka & 2023/10/23\\
    8 & R & - & 誤字の発見および説明があいまいである. & & & \yamat & 誤字の訂正と説明のニュアンスを変更. & 2023/10/21 & \yamat & 2023/10/23\\
\end{rev}
\end{document}