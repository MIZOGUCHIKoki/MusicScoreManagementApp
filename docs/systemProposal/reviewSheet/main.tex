\documentclass{reviewSheet}
\dateTime{2023/10/23 15:10-16:10}
\room{高田研究室}
\groupName{dir-en-gray}
\author{溝口 洸熙}
\object{システム提案書v2}
\member{1250297,1250352,1250341,1250372,1250373,1250382,1250385}
\newcommand{\oku}{奥平 舜理}
\newcommand{\naka}{中村 祐貴}
\newcommand{\tana}{田中 諒}
\newcommand{\mika}{三上 柊}
\newcommand{\mizo}{溝口 洸熙}
\newcommand{\yama}{山田 滉希}
\newcommand{\yamat}{山本 祥弘}

\begin{document}
\begin{rev}
	1 & R & 7/11 & 「{{削除することができる}}」は冗長な表現. & & & \naka &「{{削除することができる}}」から「削除できる」に変更. & 2023/10/21 & \mizo & 2023/10/23 \\
	2 & R & 13/1-5 & 句読点が使われている.→「.」「,」 & & \ck & \yama & 「.」「,」に変更. & 2023/10/21 & \mizo & 2023/10/23 \\
	3 & R & 13/1-5 & 文末を統一してください. & & \ck & \yama & 文末を,「だ」「である」調に変更. & 2023/10/21 & \mizo & 2023/10/23 \\
	4 & R & 13/1-5 & 一文が長すぎる & &  & \yama & 一文を短くした. & 2023/10/21 & \mizo & 2023/10/23 \\
	5 & Q & 16/6 & 「開示」は「開始」の間違いですか.& & \ck & \tana & 変更要望を\mizo に提出. & 2023/10/21 & \mika & 2023/10/10\\
	6 & Q & 13/19 & ソースは必要ないでしょうか. & & \ck & \tana & 使用したリンクを※で追加. & 2023/10/21 & \mika & 2023/10/23\\
	7 & R & 5/22 & 箇条書き\texttt{4,5}に重複項目がある. & & \ck & \mizo & \texttt{5.}を「編集」から「削除」に変更.& 2023/10/21 & \naka & 2023/10/23\\
	8 & R & 11/1-5 & 誤字の発見および説明があいまいである. & & & \yamat & 誤字の訂正と説明のニュアンスを変更. & 2023/10/21 & \yamat & 2023/10/23\\
	9 & R & 5/13 & 修正が反映されていない. & & \ck & \mizo & 修正を正しく反映させる. & 2023/10/24 & \mika & 2023/10/24 \\
	10 & R & 6/15,16 & 内容の重複 & & \ck & \mizo & 重複項目の削除  & 2023/10/24 & \mika & 2023/10/24\\
	11 & R & 5/6 & 箇条書きの改行箇所が統一されていない. & & & \mizo & 改行箇所を統一する & 2023/10/24 & \yama & 2023/10/24\\
	12 & R & 13/7 & 参照箇所が「節」ではない. & & & \mizo & 「項」とする.& 2023/10/24 & \mizo & 2023/10/24\\
	13 & R & 7/2,3 & 「ユーザ」と「ユーザー」が混在する. & & & \mizo & 「ユーザ」に統一する. & 2023/10/24 & \mika & 2023/10/24\\
	14 & R & 14/8 & 計算結果が異なる. & \ck & \ck & \tana & 正しい計算結果を記載する. & 2023/10/24 & \mika & 2023/10/24\\
\end{rev}
\end{document}