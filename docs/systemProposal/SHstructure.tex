\chapter{ハードウェア構成・ソフトウェア構成}
\section*{ハードウェア構成}
ハードウェア構成を\tblref{tbl:ハードウェア構成}に示す.
\begin{table}[p]
    \caption{ハードウェア構成}
    \label{tbl:ハードウェア構成}
    \begin{tabularx}{\textwidth}{lRr}
        \hline
        \multicolumn{1}{c}{\bfseries 項目} & \multicolumn{1}{c}{\bfseries 種類} & \multicolumn{1}{c}{\bfseries 数量} \\
        \hline
        メインサーバ                           & レンタルサーバ(Render)                  & 1                                \\
        \multirow{2}{*}{DBサーバ}           & メインサーバ上でsqlite3を利用した場合           & 0                                \\
                                         & PostgreSQLを利用した場合                & 1                                \\
        管理者端末                            & {{ブラウザが使える端末}}                   & 1                                \\
        利用者端末                            & {{ブラウザが使える端末}}                   & 1                                \\
        \hline
    \end{tabularx}
\end{table}
レンタルサーバはRender(\url{https://render.com/})を採用する.
選択した内容と金額を\tblref{tbl:レンタルサーバ}に示す.円換算は\(1\text{USD}=150\text{JPY}\)とする.
\begin{table}[p]
    \centering
    \caption{レンタルサーバの契約内容(案)}
    \label{tbl:レンタルサーバ}
    \begin{tabularx}{\textwidth}{RRrr}
        \hline
        \multicolumn{1}{c}{\bfseries 内容} & \multicolumn{1}{c}{\bfseries 詳細} & \multicolumn{1}{c}{\bfseries 表示金額(月額)} & \multicolumn{1}{c}{\bfseries 円換算(月額)} \\
        \hline
        プラン価格                            & チーム                              & 19USD                                  & 2,850JPY                              \\
        Web Services                     & starter                          & 7USD                                   & 1,050JPY                              \\
        PostgrSQL                        & starter                          & 7USD                                   & 1,050JPY                              \\
        Cron Jobs                        & starter                          & \multicolumn{1}{c}{--}                 & 1,050JPY                              \\
        \hline
    \end{tabularx}
\end{table}
\section*{ソフトウェア構成}
ソフトウェア構成を\tblref{tbl:ソフトウェア構成}に示す.
\begin{table}[p]
    \caption{ソフトウェア構成}
    \label{tbl:ソフトウェア構成}
    \begin{tabularx}{\textwidth}{RR}
        \hline
        \multicolumn{1}{c}{\bfseries 項目} & \multicolumn{1}{c}{\bfseries ソフトウェア} \\
        \hline
        フレームワーク                          & Ruby on Rails                        \\
        DBMS                             & SQLite3またはPostgreSQL                 \\
        管理者端末                            & ブラウザ                                 \\
        利用者端末                            & ブラウザ                                 \\
        \hline
    \end{tabularx}
\end{table}
データベースの複雑な設定を省くために,SQLiteを選択した.
同時接続が多くなると想定される場合にはPostgreSQLへ変更する.
以降,費用設定などの記述はPostgreSQLを利用した際の記述になる.