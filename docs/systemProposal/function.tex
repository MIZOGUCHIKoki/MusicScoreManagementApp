\chapter{機能概要・前提条件・制約事項}
\section{機能概要}
本アプリケーションへの利用登録は,管理者に対して直接行う.
管理者が認可した楽団,または個人(以下,利用者)に対してアカウントを,ID,パスワードの形式で付与する.
本サービスは,利用者に対して以下の機能を提供する.
\begin{enumerate}
    \item データベースに登録された楽譜インスタンス(以下,楽譜データ)を登録する機能.ここには,曲名,作曲者,グレード,使用楽器名などを登録できる.
    \item 登録された楽譜データを表形式で表示する機能.\\
          各項目別で並び替え機能も提供する.
    \item 登録された楽譜データを検索する機能.\\
          楽譜の属性に対して検索できる機能を提供する.
    \item 登録された楽譜データを編集する機能.
    \item 登録された楽譜データを削除する機能.
    \item リリースノートのページリンクをフッタに設置する.\\
          リリースノートとは,アプリケーションのアップデート履歴を記録するノートで,通常,不具合修正や,新機能追加の報告をユーザ向けに行う.
\end{enumerate}
\section{前提条件}
利用者が以下の条件を満足して本アプリケーションを利用することを前提としてシステムを構築する.
\begin{enumerate}
    \item 利用者がインターネットに接続でき,ブラウザを利用できる端末を所持している.
    \item 各楽団に1つのアカウントを付与し,楽団内で共有することを前提とする.つまり1つの楽団に対して複数のアカウントを発行することは想定していない.
\end{enumerate}
\section{制約事項}
本アプリケーションは,以下を制約事項とする.
\begin{enumerate}
    \item 付与したパスワードの復元機能は,ソフトウェア工学間のプロジェクト(以下,本プロジェクト)での実装予定はない.ID,パスワードの管理は利用者に一任する.万が一パスワードが漏洩した場合や利用者がパスワードを忘れた場合には,管理者の方で変更し変更後のパスワードを何らかの形で利用者に伝達する.
    \item 利用者が楽譜データの属性を追加する機能は,本プロジェクト間での実装予定はない.
\end{enumerate}