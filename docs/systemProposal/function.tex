\chapter{機能概要・前提条件・制約事項}
\section{機能概要}
データベースに登録された楽譜インスタンスを,「楽譜データ」と呼ぶ.
本サービスへの利用登録は,利用者が行う.利用者は,アカウント作成時に利用者が生成したIDとパスワードを用いて認証する.
利用者ごとに楽譜データを管理しているので,楽譜データを呼び出す際は,認可されていないアカウントからほかのアカウントが保持している楽譜データは参照できない.
\subsection{利用登録に関する機能}
\begin{enumerate}
	\item 利用登録機能.\\
	      ここでは利用登録(アカウント作成)を行う.アカウント認証はIDとパスワードを用いる.
	\item パスワード更新機能.\\
	      パスワードの管理は利用者に一任するため,パスワードを更新する機能を提供する.
	\item アカウント削除機能.\\
	      アカウントを削除する機能.
\end{enumerate}
\subsection{楽譜データ検索に関する機能}
\begin{enumerate}
	\item 楽譜データを登録する機能.ここには,曲名,作曲者,グレード,使用楽器名などを登録できる.
	\item 登録された楽譜データを表形式で表示する機能.\\
	      各項目別で並び替え機能も提供する.
	\item 登録された楽譜データを検索する機能.\\
	      楽譜の属性に対して検索できる機能を提供する.
	\item 登録された楽譜データを更新する機能.
	\item 登録された楽譜データを削除する機能.
\end{enumerate}
\subsection{運用通知に関する機能}
\begin{enumerate}
	\item リリースノートのページリンクをフッタに設置する.\\
	      リリースノートとは,アプリケーションのアップデート履歴を記録するノートで,通常,不具合修正や,新機能追加の報告をユーザ向けに行う.
\end{enumerate}
\section{前提条件}
利用者が以下の条件を満足して本アプリケーションを利用することを前提としてシステムを構築する.
\begin{enumerate}
	\item 利用者がインターネットに接続でき,ブラウザを利用できる端末を所持している.
	\item アカウントは,楽団内で共有することを前提とする.つまり1つの楽団に対して複数のアカウントを発行することは想定していない.
	\item パスワードの管理は,利用者に一任する.楽団内でパスワードを共有する場合は,パスワードを定期的に更新することを推奨する.
\end{enumerate}
\section{制約事項}
本アプリケーションは,以下を制約事項とする.
\begin{enumerate}
	\item 利用者が楽譜データの属性を追加する機能は,ソフトウェア工学間のプロジェクト(以下,本プロジェクト)間での実装予定はない.
	\item データ削除・更新後のデータ復元は受け付けない.
	\item 2年間のログインがない利用者は,その利用者のアカウントと,利用者に紐づいている楽譜データを削除する.
\end{enumerate}