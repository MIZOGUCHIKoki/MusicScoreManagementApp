\chapter{現状}
吹奏楽やオーケストラで使用するために購入する楽譜(以下,楽譜)は,大概以下のような内容物である.
\begin{itemize}
	\item スコア(全パートの譜面が載っている)
	\item 各パートの楽譜:パート数部
\end{itemize}
楽譜は,高価(5千円〜4万円超えるものも)である.
また1つの楽曲に対して必要な楽器が異なり,演奏の難易度も異なる.
\section{アンケートによる現状の把握}
楽譜管理に関する問題点を,アンケートにて調査した.アンケートは以下の人を対象とした.
\begin{itemize}
	\item 現在何らかの楽団(吹奏楽,オーケストラなど)に所属している.
	\item 過去に何らかの楽団(吹奏楽,オーケストラなど)に所属していた.
	\item 楽譜を個人的に購入している.
\end{itemize}
アンケート結果は以下のようになっている.
\begin{oframed}\label{frame:アンケート}
    \noindent{\bfseries\large アンケート結果}
    \begin{enumerate}
        \renewcommand{\labelenumi}{\textbf{\theenumi.}\ }
        \setlength{\itemsep}{.5cm}
        \item 所属していた(している)楽団の種類.\vspace{.3cm}\\
              \begin{tabularx}{\linewidth}{Rr}
                  \multicolumn{1}{c}{\bfseries 楽団の種類} & \multicolumn{1}{c}{\bfseries 投稿件数} \\
                  \hline
                  吹奏楽部                                & 21件                                \\
                  Jazz                                & 2件                                 \\
                  弦楽器のみの部活                            & 1件                                 \\
                  その他                                 & 1件                                 \\
                  \hline
              \end{tabularx}
              \clearpage
        \item 楽譜管理サービスがあるなら,使うか?(1〜5段階評価)\vspace{.3cm}\\
              \begin{tabularx}{\linewidth}{rcLL}
                  \multicolumn{2}{c}{\bfseries 評価} & \multicolumn{1}{c}{\bfseries 投稿件数} & \multicolumn{1}{c}{\bfseries 割合}        \\
                  \hline
                  使いたい                             & \textbf{5}                         & 14件                              & 56\% \\
                                                   & \textbf{4}                         & 9件                               & 36\% \\
                                                   & \textbf{3}                         & 1件                               & 5\%  \\
                                                   & \textbf{2}                         & 0件                               & 0\%  \\
                  使いたくない                           & \textbf{1}                         & 1件                               & 5\%  \\
                  \hline
              \end{tabularx}

        \item 希望するサービスの料金体制(複数回答可)\vspace{.3cm}\\
              \begin{tabularx}{\linewidth}{Rrr}
                  \multicolumn{2}{c}{\bfseries 料金体制}     & \multicolumn{1}{c}{\bfseries 投稿件数}      \\
                  \hline
                  \multirow{3}{*}{年額のサブスクリプション制(金額も含めて)} & 6,000円 / 年                         & 2件 \\
                                                         & 10,000円 / 年                        & 1件 \\
                                                         & 12,000円 / 年                        & 2件 \\
                  \hline
                  \multicolumn{2}{l}{無料で広告付き}            & 11件                                     \\
                  \multicolumn{2}{l}{買い切り型}              & 13件                                     \\
                  \hline
              \end{tabularx}
        \item 現在(もしくは以前)楽譜をどのように管理していたか? \vspace{.3cm}\\
              \begin{tabularx}{\linewidth}{Rrr}
                  \multicolumn{1}{c}{\bfseries 管理方法} & \multicolumn{1}{c}{\bfseries 投稿件数} & \multicolumn{1}{c}{\bfseries 割合} \\
                  \hline
                  紙に書いて管理している                        & 19件                                & 76\%                             \\
                  Excelなどの表計算ソフトウェアを用いている            & 3件                                 & 12\%                             \\
                  楽譜管理ソフトウェアを用いて管理している               & 1件                                 & 5\%                              \\
                  管理していない・管理せずとも問題ない                 & 2件                                 & 10\%                             \\
                  \hline
              \end{tabularx}
        \item 楽譜購入に際して問題に直面したことがあるか?\vspace{.3cm}\\
              \begin{tabularx}{\linewidth}{Rr}
                  \multicolumn{1}{c}{\bfseries 起きた問題} & \multicolumn{1}{c}{\bfseries 投稿件数} \\
                  \hline
                  楽譜の存在を確認することが{{大変}}だった              & 17件                                \\
                  \hline
              \end{tabularx}
        \item あると良い機能をあげてください.
              \begin{itemize}
                  \item 各楽譜の利用楽器別に検索したい.
                  \item 「金管5重奏」などのジャンルを記述できるような汎用的なフィールドが欲しい.
                  \item 各楽譜の出版社がわかると探しやすい.
                  \item どこからでもアクセスしたい.
                  \item グレード別検索ができるようにしてほしい.
                  \item 作曲者・編曲者,演奏時間も知りたい.
                  \item 参考音源のURLを貼れる場所が欲しい.
              \end{itemize}
    \end{enumerate}
\end{oframed}
\section{アンケート結果を踏まえて}\label{chap:issues}
アンケート結果を踏まえて現状を整理する.全回答数25件のうち,紙に書いて楽譜を管理していると回答した人はおよそ80\%,
さらに楽譜管理サービスを提供すると使用するかという質問に対しては,90\%の人が使いたいと投票している.\par
楽譜購入の際に起きた問題については,「購入する前に,その楽譜を所有しているかの確認が{{大変}}であった」という回答が全回答数の85\%を占めており,
楽譜管理サービスを提供するうえであると良い機能を募集したところさまざまな意見が挙がった.\par
これらの問題を解決し期待に応えるべく,第\ref{chap:課題解決のための提案}項では大まかな解決方法を提案する.