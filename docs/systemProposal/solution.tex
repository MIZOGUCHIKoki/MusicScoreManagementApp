\chapter{課題解決のための提案}\label{chap:課題解決のための提案}
\section{予備知識}
楽譜には,曲名・難易度・必要楽器などのさまざまな要素が含まれている.
また,楽譜には著作権が存在するため,楽譜そのものを画像,またはデータ形式等にして管理できない.
そのため,本アプリケーションでは楽譜そのものでなく,その存在を扱っていくことになる.
\section{現状の課題}
課題として,「紙で楽譜を管理しているため,存在を確認することが容易でなく,また楽曲名以外の情報から絞り込んで探すことも困難である」ことが挙げられる.
\section{提案}
課題の解決のため,「楽譜をさまざまな要素とともにデータベースに登録・削除でき,またそれを用いた検索によりその楽譜を所有しているかを容易に確認できる」システムを提案する.
これを利用することにより,「同一楽譜を購入してしまう」「ある楽曲にどれだけの人数が必要かすぐにわからない」といった問題を解決できる.\par
システムのイメージとして,蔵書検索の楽譜管理版を考えている.