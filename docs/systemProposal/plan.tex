\chapter{開発体制と工程計画}
開発は,本チームメンバ7名で行う.
ソースコードの共有やバージョン管理にはGitHubを用いる.
工程計画を\tblref{tbl:工程計画}に示す.
\begin{table}[h]
	\centering
	\caption{工程計画}
	\label{tbl:工程計画}
	\renewcommand{\arraystretch}{1.5}
	\begin{tabularx}{\textwidth}{|l|r|r|C|C|C|C|C|C|C|C|}
		\hline
		\multicolumn{1}{|c|}{\bfseries 工程過程} & \multicolumn{1}{c|}{\bfseries 開始日} & \multicolumn{1}{c|}{\bfseries 終了日} & \multicolumn{2}{c|}{2023.10} & \multicolumn{2}{c|}{2023.11} & \multicolumn{2}{c|}{2023.12} & \multicolumn{2}{c|}{2024.1}                     \\
		\hline
		要求分析                                 & 10/2                               & 10/29                              & 2                            & 29                           &                              &                             &    &    &    &    \\
		\hline
		外部設計                                 & 10/30                              & 11/26                              &                              & 30                           &                              & 26                          &    &    &    &    \\
		\hline
		内部設計                                 & 11/27                              & 12/14                              &                              &                              &                              & 27                          & 14 &    &    &    \\
		\hline
		コーディングテスト・単体テスト                      & 12/18                              & 1/4                                &                              &                              &                              &                             &    & 18 & 4  &    \\
		\hline
		結合テスト・総合テスト                          & 1/8                                & 1/18                               &                              &                              &                              &                             &    &    & 8  & 18 \\
		\hline
		納品                                   & \multicolumn{2}{c|}{1/29}          &                                    &                              &                              &                              &                             &    &    & 29      \\
		\hline
	\end{tabularx}
\end{table}