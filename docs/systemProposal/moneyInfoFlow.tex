\chapter{情報・金銭の流れ}
\section{情報の流れ}\label{sec:情報の流れ}
構築するシステムにおける情報の流れは以下の通りである(\figref{fig:情報の流れ}).
ある楽団に所属するユーザーが,楽譜の情報をデータベース登録する.
同じ楽団に所属するほかのユーザーが,IDとパスワードを用いてサービスにログインし,楽譜を検索する.
検索の際に,登録されている楽譜データの属性に対して検索できるようにする.\par
また,管理者は,サーバのユーザ情報が格納されているDB(以下,ユーザDB)に直接アクセスでき,任意のユーザに対してパスワードを更新できる.
管理者は,楽譜の情報が格納されているDB(以下,楽譜DB)は操作しない.
\begin{figure}[h]
    \centering
    \begin{tikzpicture}
	\newcommand{\ic}[1]{\includegraphics[keepaspectratio,width=3cm]{#1}}
	\node (ms-db){\ic{../network_iconset/server_db.pdf}};
	\node at(ms-db.north){楽譜DB};
	\node [right=10cm of ms-db](usr-db){\ic{../network_iconset/server_db.pdf}};
	\node at(usr-db.north){ユーザDB};
	\node at($($(ms-db)!0.5!(usr-db)$)+(0,-5.5cm)$)(server){\ic{../network_iconset/server}};
	\node at(server.south){Webサーバ};
	\node at(usr-db|-server)(user){\ic{../network_iconset/network_user_w.pdf}};
	\node at(user.south){利用者};
	\node at($(ms-db)!0.5!(usr-db)$)(admin){\ic{../network_iconset/network_administrator.pdf}};
	\node at(admin.north){管理者};

	% Arrow
	\newcommand{\col}{green!50!black}
	\newcommand{\coll}{red!70!black}
	\draw[-latex,very thick] (admin.south)--(server.north)node[midway,left]{\scriptsize 運用・広告更新};
	\draw[-latex,very thick,blue] ($(user.west)+(0,.9cm)$)--($(server.east)+(0,.9cm)$)node[midway,above]{\scriptsize 認証情報・利用者情報};
	\draw[-latex,very thick,blue] ($(server.east)+(0,.3cm)$)--($(user.west)+(0,.3cm)$)node[midway,above]{\scriptsize 認証応答・結果表示};

	\draw[latex-latex,very thick,blue] ($(server.north east)+(0,-.3cm)$)--($(usr-db.south west)+(.3cm,0)$)node[midway,below right]{\scriptsize 登録情報クエリ・結果};

	\draw[\col,-latex,very thick] ($(user.west)+(0,-.3cm)$)--($(server.east)+(0,-.3cm)$)node[midway,below]{\scriptsize 登録・検索・削除など};
	\draw[\col,-latex,very thick] ($(server.east)+(0,-.9cm)$)--($(user.west)+(0,-.9cm)$)node[midway,below]{\scriptsize 結果表示};

	\draw[\col,-latex,very thick] ($(server.west)+(0,.3cm)$)-|($(ms-db.south)+(.3cm,0)$)node[midway,above right]{\scriptsize クエリ(ユーザ情報を探索条件に含む)};
	\draw[\col,-latex,very thick] ($(ms-db.south)+(-.3cm,0)$)|-($(server.west)+(0,-.3cm)$)node[midway,below right]{\scriptsize 結果};

	\draw[-latex,very thick] (admin.east)--(usr-db.west)node[midway,above]{\scriptsize 運用上のアクセス};
	\draw[-latex,very thick] (admin.west)--(ms-db.east)node[midway,above]{\scriptsize 運用上のアクセス};

	\draw[\coll,-latex,very thick] (server.north west)--(ms-db.south east)node[midway,below left]{\scriptsize 楽譜データの削除};
	\draw[\coll,-latex,very thick] ($(server.north east)+(-.3cm,0)$)--($(usr-db.south west)+(0,.3cm)$)node[midway,above left]{\scriptsize 利用者データの削除};

	\draw[\coll,-latex,very thick] ($(ms-db|-server)+(0,-2.5cm)$)--($(ms-db|-server)+(2cm,-2.5cm)$)node[right,text=black]{\tiny 自動処理};
	\draw[-latex,very thick] ($(ms-db|-server)+(0,-3cm)$)--($(ms-db|-server)+(2cm,-3cm)$)node[right,text=black]{\tiny 管理者からのアクセス};
	\draw[\col,-latex,very thick] ($(ms-db|-server)+(5cm,-2.5cm)$)--($(ms-db|-server)+(7cm,-2.5cm)$)node[right, text=black]{\tiny 認可された利用者が承認された部分でのアクセス};
	\draw[blue,-latex,very thick] ($(ms-db|-server)+(5cm,-3cm)$)--($(ms-db|-server)+(7cm,-3cm)$)node[right, text=black]{\tiny 利用者情報処理・認証処理};
\end{tikzpicture}
    \caption{情報の流れ}
    \label{fig:情報の流れ}
\end{figure}
\section{金銭の流れ}\label{sec:金銭の流れ}
金銭の流れを\figref{fig:金銭の流れ}に示す.
\begin{figure}[h]
    \centering
    \begin{tikzpicture}
    \newcommand{\ic}[1]{\includegraphics[keepaspectratio,width=3cm]{#1}}
    \node(admin){\ic{../network_iconset/network_administrator.pdf}};
    \node at(admin.north){管理者};
    \node[right=6cm of admin](ads){\ic{../network_iconset/network_user.pdf}};
    \node at(ads.north){広告依頼者};
    \node[below=4cm of ads](usr){\ic{../network_iconset/network_user_w.pdf}};
    \node at(usr.south){利用者};
    \node at(admin|-usr)(server){\ic{../network_iconset/virtual_server.pdf}};
    \node at(server.south){レンタルサーバ};


    \draw[-latex,very thick](ads.west)--(admin.east)node[midway,fill=white,draw]{広告費用};
    \draw[-latex,very thick](usr.north)--(ads.south)node[midway,fill=white,draw]{商品購入};
    \draw[-latex,dashed,very thick](admin.south east)--(usr.north west)node[midway,fill=white,draw]{広告掲示};
    \draw[-latex,very thick](admin.south)--(server.north)node[midway,fill=white,draw]{レンタルサーバ代};
\end{tikzpicture}
    \caption{金銭の流れ}
    \label{fig:金銭の流れ}
\end{figure}

