\chapter{情報・金銭の流れ}
\section*{情報の流れ}\label{sec:情報の流れ}
構築するシステムにおける情報の流れは以下の通りである(\figref{fig:情報の流れ}).
ある楽団に所属するユーザが,楽譜の情報をデータベース登録する.
同じ楽団に所属するほかのユーザが,IDとパスワードを用いてサービスにログインし,楽譜を検索する.
検索の際に,登録されている楽譜データの属性に対して検索できるようにする.\par
また,管理者は,サーバのユーザ情報が格納されているデータベース(以下,ユーザDB)に直接アクセスでき,任意のユーザに対してパスワードを更新できる.
楽譜データが格納されているデータベース(以下,楽譜DB)の容量を確保するため,Webサーバが2年間サービスを利用していない楽団の楽譜DBにアクセスしデータを自動的に削除する.
\begin{figure}[p]
    \centering
    \begin{tikzpicture}
    \newcommand{\ic}[1]{\includegraphics[keepaspectratio,width=3cm]{#1}}
    \node(server){\ic{../network_iconset/server}};
    \node at(server.south){Webサーバ};
    \node[above=3cm of server](ms-db){\ic{../network_iconset/server_db.pdf}};
    \node at(ms-db.north){楽譜DB};
    \node[left=3.5cm of ms-db](usr-db){\ic{../network_iconset/server_db.pdf}};
    \node at(usr-db.north){ユーザDB};
    \node[right=3.5cm of ms-db](dbb){\ic{../network_iconset/server_db.pdf}};
    \node at(dbb.north){バックアップ};
    \node at(dbb|-server)(user){\ic{../network_iconset/network_user_w.pdf}};
    \node at(user.south){利用者};
    \node at(usr-db|-server)(admin){\ic{../network_iconset/network_administrator.pdf}};
    \node at(admin.south){管理者};

    % Arrow
    \draw[-latex,very thick] (admin.east)--(server.west)node[midway,above]{\scriptsize 運用・広告更新};
    \draw[-latex,very thick,blue] ($(user.west)+(0,.9cm)$)--($(server.east)+(0,.9cm)$)node[midway,above]{\scriptsize 認証情報};
    \draw[-latex,very thick,blue] ($(server.east)+(0,.3cm)$)--($(user.west)+(0,.3cm)$)node[midway,above]{\scriptsize 認証応答};

    \draw[latex-latex,very thick,blue] (server.north west)--(usr-db.south east)node[midway,above,fill=white,draw=blue]{\scriptsize 認証情報照会};

    \draw[-latex,very thick] ($(user.west)+(0,-.3cm)$)--($(server.east)+(0,-.3cm)$)node[midway,below]{\scriptsize 検索};
    \draw[-latex,very thick] ($(server.east)+(0,-.9cm)$)--($(user.west)+(0,-.9cm)$)node[midway,below]{\scriptsize 結果表示};

    \draw[-latex,very thick] ($(server.north)+(.3cm,0)$)--($(ms-db.south)+(.3cm,0)$)node[midway,right]{\scriptsize クエリ};
    \draw[-latex,very thick] ($(ms-db.south)+(-.3cm,0)$)--($(server.north)+(-.3cm,0)$)node[midway,left]{\scriptsize 結果};

    \draw[Stealth-Stealth, ultra thick,red](ms-db.west)--(usr-db.east)node[midway,above]{\scriptsize リレーション};

    \draw[-latex,very thick] (admin.north)--(usr-db.south)node[midway,fill=white,draw]{\scriptsize 利用者情報の登録/更新など};

    \draw[-latex,thick,dashed] ($(usr-db.north)+(0,.2cm)$)|-($(dbb.north)+(0,.7cm)$)--($(dbb.north)+(0,.2cm)$);
    \draw[thick,dashed] ($(ms-db.north)+(0,.2cm)$)--($(ms-db.north)+(0,.7cm)$)node[fill=white,draw,dashed]{\scriptsize DBの定期バックアップ};
\end{tikzpicture}
    \caption{情報の流れ}
    \label{fig:情報の流れ}
\end{figure}
\section*{金銭の流れ}\label{sec:金銭の流れ}
金銭の流れを\figref{fig:金銭の流れ}に示す.
広告依頼者は,管理者に広告費用として1広告あたりの金額を,銀行振込などの方法で支払う.
管理者はレンタルサーバ代を支払い,自身の管理費を差し引く.
利用者は広告次第では広告先の商品を購入することになるので,利用者から広告依頼者へお金が渡る.
\begin{figure}[p]
    \centering
    \begin{tikzpicture}
	\newcommand{\ic}[1]{\includegraphics[keepaspectratio,width=3cm]{#1}}
	\node(admin){\ic{../network_iconset/network_administrator.pdf}};
	\node at(admin.north){管理者};
	\node[right=6cm of admin](ads){\ic{../network_iconset/network_user.pdf}};
	\node at(ads.north){広告依頼者};
	\node[below=4cm of ads](usr){\ic{../network_iconset/network_user_w.pdf}};
	\node at(usr.south){利用者};
	\node at(admin|-usr)(server){\ic{../network_iconset/virtual_server.pdf}};
	\node at(server.south){レンタルサーバ};


	\draw[-latex,very thick](ads.west)--(admin.east)node[midway,fill=white,draw]{広告費用};
	\draw[-latex,very thick](usr.north)--(ads.south)node[midway,fill=white,draw]{商品購入};
	\draw[-latex,dashed,very thick](admin.south east)--(usr.north west)node[midway,fill=white,draw]{広告掲示};
	\draw[-latex,very thick](admin.south)--(server.north)node[midway,fill=white,draw]{レンタルサーバ代};
\end{tikzpicture}
    \caption{金銭の流れ}
    \label{fig:金銭の流れ}
\end{figure}