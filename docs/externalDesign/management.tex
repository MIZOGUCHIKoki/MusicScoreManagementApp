\chapter{運用保守}
\section{運用}
運用の項目として以下のものが考えられる.
\begin{itemize}
    \item 重大な問題の発生時には管理者によりサービスの停止・再開を制御する.
    \item システムの更新時には,インシデントに対して復元を可能にするためのデータを保存する.
    \item 運用終了は,収支などを理由としての終了が予想される.
\end{itemize}
ソースコードの管理はGitHubで行う.CIツールとしてGitHub Actionsを用い,各ブランチへのプッシュの際に,\texttt{rails test}を実行する.
また,ソースコードの可読性を向上させるため,同時に\texttt{rubocop}を実行し,Rubyコードの表記が統一されていない場合は,CIをFailedとする.\par
アプリケーションバージョンはGitTagで管理し,Tagが発行された場合に,対象コミットを本番環境へデプロイするCDも,GitHub Actionsが担う.
開発ブランチを\texttt{develop}と定め,デフォルトブランチをこれにする.
\section{保守}
保守の項目として以下のものが考えられる.
\begin{itemize}
    \item サーバの運用および保守はレンタルサーバの提供会社に委託する.
    \item バグ報告や機能改善要望はメール窓口により受け付ける.
    \item セキュリティ向上のために,パッケージなどの更新をする.
\end{itemize}
\subsection*{メール窓口について}
メールについては,GoogleやYahoo!などのフリーメールアドレスを作成し,受信したメッセージに対応する.
メールアドレスを記載したお問い合わせページを,全ページ共通のヘッダに記載する.
\subsection*{バックアップ}
機能改善などに伴う新システムの適用時には,適用前のシステムをバックアップとして管理者が保存しておく.
このバックアップデータは,新システム提供時にバグが発見された際,旧システムに復元することを目的としている.
なお,データベースサーバの保守に関しては,レンタルサーバ会社に一任する.
\section{非機能要件}
管理者としてシステムにアクセスできる人物は,dir-en-grayの役員7名とする.なお,管理者はセキュリティポリシーを遵守するものとする.\par
パスワードのブルートフォースアタックについては,Railsの\texttt{rack-attack}を用いて対策する.
また,漏洩したパスワードを設定しないよう促す,Railsの\texttt{pwne}を導入する.
さらに,GitHubの\texttt{dependabot}を用いて,\texttt{gem}経由で適用しているパッケージのバージョンはできるだけ最新版に更新する.\par
障害検知は,AWSの\texttt{StatusCheckFailed\_System}を用いる.障害時は管理者が障害原因を究明し,速やかに対処する.
アップデートにより生じた障害については,一時的にバージョンを落とすことも検討する.