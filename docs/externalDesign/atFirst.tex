\chapter{{はじめに}}
\section{課題}
楽譜は高価であり,大規模な楽団だと数百という数を所有していることもある.
そんな中で,新しい楽譜を購入するとなった際に,楽団の所有している楽譜が把握できていないと金銭と時間を浪費してしまうことになりかねない.
また,特定の楽器が含まれた楽譜を探したり,難易度別に楽譜を見繕いたいといった,楽譜に含まれた情報をうまく活用したい場面もでてくる.
しかし,現在はそういったことをするにも紙やExcelで管理し,手間と時間がかかっている状態である.
楽譜管理に関する問題点を以下の人を対象として調査した.
\begin{itemize}
    \item 現在何らかの楽団へ所属している.
    \item 過去に何らかの楽団へ所属していた.
    \item 楽譜を個人的に購入している.
\end{itemize}
調査の結果,全回答数25件のうち,紙で管理している人が80\%,
さらに楽譜管理サービスを使用するかという質問に対しては90\%が使いたいと回答した.
\section{機能概要}
ユーザ側の機能としては以下の通りである.
\begin{itemize}
    \item アカウントの作成,削除,編集
    \item 楽譜データの登録,検索,削除,編集
    \item ログイン
    \item ログアウト
\end{itemize}
管理者側の機能としては以下の通りである.
\begin{itemize}
    \item 管理者アカウント編集
    \item ユーザ情報の編集,削除
    \item 楽譜データの削除
    \item 広告更新
    \item ログイン
    \item ログアウト
\end{itemize}
アカウントが2年間使用されていない場合,Webサーバによりアカウントと楽譜データは自動で削除される.
\begin{figure}[h]
    \centering
    \input{tikz/informationFlow.tex}
    \caption{情報の流れ}
\end{figure}
\section{利用の流れ}
ユーザはアカウントを作成し,ログインを行う.
ログイン後はトップページに遷移し,楽譜データの登録,検索,削除,編集ができる.
また,トップページ画面からマイページ画面へ遷移すると,アカウントの削除とアカウント情報の変更,ログアウトが可能である.\par
管理者は,ユーザと同様にログイン画面でIDとパスワードを入力することにより管理者専用のページにログインを行う.
管理者専用ページからはユーザの削除,編集と楽譜データを削除できる.