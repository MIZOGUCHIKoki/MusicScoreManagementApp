\documentclass{reviewSheet}
\dateTime{2023/11/13 14:40-16:10}
\room{A503}
\groupName{dir-en-gray}
\author{溝口 洸熙}
\object{外部設計書 v1}
\member{1250297,1250352,1250341,1250372,1250373,1250382,1250385}
\newcommand{\oku}{奥平 舜理}
\newcommand{\naka}{中村 祐貴}
\newcommand{\tana}{田中 諒}
\newcommand{\mika}{三上 柊}
\newcommand{\mizo}{溝口 洸熙}
\newcommand{\yama}{山田 滉希}
\newcommand{\yamat}{山本 祥弘}

\begin{document}
\begin{rev}
1 & R & 2/10 & 文章がおかしい &  &  & \mizo  & 同様”に”ログイン画面”で” & 2023/11/13 & \mika  &  \\
2 & R & 2/9 & 助詞の使い方がおかしい. &  &  & \mizo  & 楽譜データ”を”変更する & 2023/11/13 & \mizo  &  \\
3 & R & 3/4 & 消費者メールアドレスではない &  &  & \mizo  & ユーザメールアドレスに変更 & 2023/11/13 & \mika  &  \\
4 & R & 3/5 & カウントを登録できる &  &  & \mizo  & アカウントを登録する. & 2023/11/13 & \mizo  &  \\
5 & R & 3/18 & 登録データの中に編曲者が入っていない &  &  & \mizo  & 編曲者を追加する. & 2023/11/13 & \mika  &  \\
6 &  & 2/5 & {{管理者アカウントはひとつ}} &  &  & \mizo  & {{管理者用アカウントは1つであるという旨を追加}} & 2023/11/13 & \oku  &  \\
7 &  & 2/18 & ログインのほうに管理者についても書きたい &  &  & \mizo  & ログインの方に管理者についても書く. & 2023/11/13 & \oku  &  \\
8 & R & 5/4 & ユーザ名がない &  &  & \mizo  & アカウント情報でユーザ名を追加する & 2023/11/13 & \tana  &  \\
9 & R & 7/2 & 図3-2の見出し,管理者側画面遷移図ではない &  &  & \mizo  & 図3-2管理者側画面遷移図→利用者側画面遷移図 & 2023/11/13 & \naka  &  \\
10 & R & 7/図3-1,3-2 & 画面遷移図の視認性が悪い &  &  & \mizo  & 画面遷移図2つを2段に変更 & 2023/11/13 & \yamat  &  \\
11 & R & 5/8 & パスワードが抜けている.文章がおかしい &  &  & \mizo  & アカウント機能のパスワードが抜けている.また文章が変更の旨が2回出てきており不適切. &  & \tana  &  \\
12 &  & 5/1 & 図が見にくい &  &  & \mizo  & 分割しない & 2023/11/13 & \yamat  &  \\
13 & R & 6/図2-7 & 編集機能であるのに終わりが削除となっている &  & \ck & \tana  & 削除→編集に変更する & 2023/11/13 & \mika  &  \\
14 & R & 4/19,21 & ユーザではわかりづらい &  &  & \mizo  & ユーザの中に管理者が含まれることを明記 & 2023/11/13 & \mika  &  \\
15 & R & 6/図2-9 & キャプションが枠外に出ていない &  & \ck & \mizo  & キャプションを枠外に出す & 2023/11/13 & \mika  &  \\
16 & R & 8/図3-3〜23 & UI Viewの視認性が悪い &  &  & \mizo  & 図の分割数を調整する & 2023/11/13 & \yamat  &  \\
17 & R & 6/図2-13 & タイプミス &  &  & \mizo  & タイプミスを修正する & 2023/11/13 & \mika  &  \\
18 & R & 7/図3-2 & マイページからトップページへ遷移できない &  & \ck & \tana  & 両矢印に変更する & 2023/11/13 & \mika R &  \\
19 & R & 10/8 & 目的語がない. &  &  & \mizo  & “CIを”を追加する. & 2023/11/13 &  &  \\
20 & R & 6/図2-10 & 機能名が間違っている &  & \ck & \tana  & 「ユーザ編集機能」を「管理者情報編集機能」に変更する & 2023/11/13 & \yama  &  \\
21 & R & 6/10 & 説明を追加 &  &  & \mizo  & ユーザを削除した際に,関連した楽譜データを削除する旨を追加する & 2023/11/13 & \tana  &  \\
22 &  &  & 至るところに利用者という文字があった. &  &  &  & ユーザに変更 & 2023/11/13 & \oku  &  \\
23 & R & 11/8 & ピリオドがカンマになっている &  &  & \mizo  & カンマをピリオドに変更 & 2023/11/13 & \mizo  &  \\
24 &  &  & 接続詞が足りない &  &  & \mizo  & ”さらに”追加. & 2023/11/13 & \mizo  &  \\
25 & R & 1/5 & 接続詞が不適切である &  &  & \mizo  & ”したり”を”し,”に変更 & 2023/11/13 & \yamat  &  \\
26 & R & 5/図2-2 & 必要のない↓がある &  & \ck &  \yama  & 必要のない↓を削除する &  & \yama  &  \\
27 & R & 1/4 & 文章がおかしい &  &  & \mizo  & 欲しい→探す,見繕ったりしたい→「見繕いたい」に変更 & 2023/11/13 & \mika  &  \\
28 & R & 3/4 & 文章がおかしい &  &  & \mizo  & 文章の校正 & 2023/11/13 & \mika  &  \\
29 & R & 3/21 & 文章統一 &  &  & \mizo  & DBという文言を,ユーザ情報の編集,削除という文体に統一 & 2023/11/13 & \tana  &  \\
30 & R & 3/22 & 文章統一 &  &  & \mizo  & DBという文言を,ユーザ情報の編集,削除という文体に統一 & 2023/11/13 & \tana  &  \\
31 & R & 11/4-5 & データベースの保守→データベースサーバの保守 &  &  & \mizo  & 指摘を適用する & 2023/11/13 & \mizo  &  \\
\end{rev}
\end{document}