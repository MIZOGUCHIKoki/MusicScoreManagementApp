\documentclass{reviewSheet}
\dateTime{2023/11/14 14:40-16:10}
\room{A559}
\groupName{dir-en-gray}
\author{溝口 洸熙}
\object{外部設計書 v1}
\member{1250297,1250352,1250341,1250372,1250373,1250382,1250385}
\newcommand{\oku}{奥平 舜理}
\newcommand{\naka}{中村 祐貴}
\newcommand{\tana}{田中 諒}
\newcommand{\mika}{三上 柊}
\newcommand{\mizo}{溝口 洸熙}
\newcommand{\yama}{山田 滉希}
\newcommand{\yamat}{山本 祥弘}

\begin{document}
\begin{rev}
    1 & R & 1/2 & 文章が話し言葉である &  &  & \mizo  & 「そんな中で,」を書き言葉に変更 & 2023/11/14 & \yamat   & 2023/11/14 \bk
    2 & R & 2/10 & ユーザIDではなく,メールアドレス &  &  & \mizo  & 「メールアドレス」に変更 & 2023/11/14 & \mizo  &  2023/11/14\bk
    3 &  & 3/5 & メールアドレスでユーザを識別する旨を &  &  & \mizo  & 指摘点を適用 & 2023/11/14 & \mizo  & 2023/11/14 \bk
    4 & R & 図3-3,4,11,19 & 不要な情報が含まれている &  & \ck & \mika  & 不要な情報を削除 & 2023/11/14 & \mika  & 2023/11/14 \bk
    5 & R & 25/8 & 文章中でカンマがピリオドになっている &  &  & \mizo  & ピリオドをカンマに修正 & 2023/11/14 & \yamat  & 2023/11/14 \bk
    6 & R & 24/9 & バージョン例がわかりづらい &  &  & \mizo  & わかりよく変更 & 2023/11/14 & \mika  & 2023/11/14 \bk
    7 & R & 24/19 & {{文章がおかしい}} &  &  & \mizo  & 「対して対応」→「対応」に変更 & 2023/11/14 & \mika  & 2023/11/14  \bk
    8 & R & 内容 & 管理者のメールアドレスをどう識別させるか & \ck & \ck & \mizo  & AdminInfo DBを追加する. & 2023/11/14 & \oku  & 2023/11/14 \bk
    9 & R & 3/22,4/2 & {{文章がおかしい}} &  &  & \mizo  & 「登録されている曲から任意の曲」→「登録されている任意の曲」に変更 & 2023/11/14 & \naka  & 2023/11/14 \bk
    10 & R & 4/最終 & 処理内容が本来の位置でない & \ck & \ck & \mizo  & 処理内容を本来の位置に変更 & 2023/11/14 & \mika & 2023/11/14\bk
\end{rev}
\end{document}